Digilent provides a branch of the linux kernel that is compatible with most of
their boards, including the ZedBoard~\cite{DigilentLinux}.
Source revision \emph{06b3889} was found to be the most recent to include
android support.

To build linux from~\cite{DigilentLinux} for the ZedBoard with Android support,
the following steps need to be followed:
\begin{enumerate}
	\item \emph{Fix the device tree} by copying the device tree from
		\cite{DigilentReferenceDesign} to
		\emph{<kernel>/arch/arm/boot/dts/digilent-zed.dts}
	\item \emph{Setup build environment to use Xilinx tools} by setting
		environment variables as seen in \Cref{lst:envsetup}
	\item \emph{Generate default configuration for the Zedboard} by running
		\emph{make digilent\_zed\_defconfig}
	\item \emph{Add Android and touchscreen related configurations} by running
		\emph{make menuconfig}
		\begin{enumerate}
			\item Navigate to `Device Drivers' and enable `Staging Drivers'
			\item Navigate to `Device Drivers' `Staging Drivers', `Android'
				and enable all entries
			\item Navigate to `Device drivers', `Input device support' and
				enable `Touchscreens'
			\item Navigate to `Device drivers', `Input device support',
				`Touchscreens' and enable
				\begin{itemize}
					\item `Ilitek ILI210X based touchscreen'
					\item `USB Touchscreen Driver'
				\end{itemize}
			\item Navigate to `Device drivers', `HID support', `Special HID
				drivers' and enable `HID Multitouch panels'
		\end{enumerate}
	\item \emph{Build the kernel} by running \emph{make}
	\item \emph{Build the device tree} by running \emph{./scripts/dtc/dtc -I dts
		-O dtb -o devicetree.dtb arch/arm/boot/dts/digilent-zed.dts}
\end{enumerate}
\begin{lstlisting}[
	language=Bash,
	caption={Environment setup to build the linux kernel},
	label={lst:envsetup},
	basicstyle=\small,
	float=h,
	floatplacement=h
	]
export CCOMPILER=arm-xilinx-linux-gnueabi-gcc
export ARCH=arm
export CROSS_COMPILE=arm-xilinx-linux-gnueabi-
export PATH=$PATH:/opt/Xilinx/14.7/ISE_DS/EDK/gnu/arm/lin/bin/		
\end{lstlisting}
The compiled image will be available in \emph{arch/arm/boot/zImage}.
To boot it requires a ramdisk, available from~\cite{DigilentReferenceDesign}.
To have the linux kernel to load our kernel modules and start Android when the
boot was successul, the ramdisk has to be adapted.
The easiest way to do this is to make the startup script in
\emph{etc/init.d/rcS} execute our own startup scripts from the SD card after
everything else is done.

We created two startup scripts, one to load the kernel modules
(\emph{<repo>/linux-files/startup.sh}) and one to start android
(\emph{<repo>/linux-files/startup\_android.sh}).
