The branch of the Linux kernel provided by Digilent \cite{DigilentLinux} already included a device driver for partial reconfiguration.

Before the driver can be used, it must be set up during the boot routine. \Cref{lst:setuppr} shows how to set up the device driver. In \Cref{lst:ispart} the flag for partial reconfiguration is set.

\begin{lstlisting}[
language=Bash,
caption={Setup partial reconfiguration driver},
label={lst:setuppr},
basicstyle=\small,
float=h,
floatplacement=h
]
echo "++ Adding partial reconfiguration device node"
mknod /dev/xdevcfg c 259 0
echo "1" > /sys/devices/axi.0/f8007000.devcfg/is_partial_bitstream(*@\label{lst:ispart}@*)
\end{lstlisting}

From Android, \gls{dpr} can be easily done by sending the path of the partial bitstream to the device driver. \Cref{lst:setuppr} shows how this is implemented.

\begin{lstlisting}[
language=Bash,
caption={Partial reconfiguration out of Android},
label={lst:setuppr},
basicstyle=\small,
float=h,
floatplacement=h
]
cat new_bitstream.bin > /dev/xdevcfg
\end{lstlisting}

After the reconfiguration is completed, the new logic part can be used.
