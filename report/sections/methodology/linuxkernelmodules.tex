
\textbf{Blake2b Driver:}\\
The blake2b device driver sits between the userspace application that desires to
compute the hash of a file and the \gls{pl} implementation of Blake2B.\\
The communication with the \gls{pl} was discussed in detail in
\Cref{sssec:blake2bmodule}.\\
The interface to the userspace application is simple.
The driver creates a device file \emph{/proc/blake2b}.
The userspace application can then write a file path to that file.
This triggers the start of the hashing function.
Afterwards, the application can read $64$ bytes from the device file, these
bytes will contain the hash.\\
The blake2b driver code is available at \emph{<repo>/drivers/blake2b/blake2b.c}.
The code is quite self explanatory and therefore not discussed in detail here.
One caveat is that one has to call the function \emph{wmb} between subsequent
reads or writes to the device.
This function will impede the compiler to rearrange the calls during
optimization and therefore guarantees that the reads and writes are executed in
specified order.
\\\\
\textbf{Image Filter Driver:}\\
The Image Filter device driver writes the to process raw \gls{rgb} data to the write register address of the programmed filter logic. The processing of the value is then triggered automatically. Afterwards the filter data is read from the read register address by the driver. The interface to the userspace application is implemented by writing and reading from a related \emph{.bin} file. The absolute path of this file has to be communicated to the device driver.\\
The communication with the \gls{pl} was discussed in section \ref{sssec:imageprocessingmodule}. The communication with the application was discussed in section \ref{sssec:imageprocessingapp}. The developed driver creates a device file \emph{/proc/simple_filters}, which can be triggered by the user application. The driver code itself is available at \emph{<repo>/drivers/simple_filters/simple_filters.c}.
