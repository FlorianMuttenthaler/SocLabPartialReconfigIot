The Zynq Software Design consists of the following components:
\begin{itemize}
	\item The \gls{fsbl}
	\item The \emph{u-boot} utility
	\item A linux kernel
	\item The Android Operating System
	\item Linux kernel modules
	\item The image processing app
\end{itemize}
Setting up the \gls{fsbl} and the u-boot utility is described in detail
in~\cite{DigilentTutorial}.
The only important thing to note is that u-boot needs to be built from source
code revision \emph{b55d4b1}, as the support for the ZedBoard changed
afterwards.
\cite{DigilentTutorial} also provides instructions on how to build the linux
kernel, but since we need it to be able to boot Android quite some modifications
were needed so it is explained in detail in \Cref{sssec:linuxonzedboard}.
The Android setup itself is discussed in \Cref{sssec:androidonzedboard}, while
\Cref{sssec:linuxkernelmodules} talks about the kernel modules and
\Cref{sssec:imageprocessingapp} details the Android app.
Finally, \Cref{sssec:dynamicpartialreconfiguration} talks about how we set up
the software system to be able to dRo \gls{dpr}.
