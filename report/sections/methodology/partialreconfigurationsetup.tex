Partial reconfiguration was done using the planAhead tool included in the Xilinx ISE Design Suite. Compared to the Vivado Design Suite, where \gls{dpr} works with the \gls{vhdl} files, the planAhead tool only works with the synthesized netlists.

For \gls{dpr} a Bottom-Up Synthesis is required. The static logic is synthesized with a black box module definition for each \gls{dpr} module.

The synthesis is done inside the script \emph{<repo>/bootimage/generate\_without\_android.sh}. Since the netlists for the different filter logics are not synthesized with the flow, they must be generated separately. Both is done with the following lines: 

\begin{lstlisting}[
	language=Bash,
	caption={Synthesis for project},
	label={lst:makenetlist},
	basicstyle=\small,
	float=h,
	floatplacement=h
	]
make -f system.make netlist
# generate netlists for filter logic
xst -ifn synth_filter_logic.xst
\end{lstlisting}

Xilinx provides a tutorial \cite{planAheadTutorial} that describes how the \gls{dpr} is done with the planAhead tool. First, the \gls{gui} based design was used to create the bitstreams. Afterwards a \gls{tcl} script, \emph{<repo>/hardware_design/planAhead.tcl}, was implemented with the included \gls{tcl} console. This script automatically generates the bitstreams used for partial reconfiguration. 

In our project $6$ bitstreams are generated:
\begin{itemize}
	\item full bitstream with red filter
	\item full bitstream with green filter
	\item full bitstream with blue filter
	\item partial bitstream with red filter
	\item partial bitstream with green filter
	\item partial bitstream with blue filter
\end{itemize}
